\documentclass[a4paper]{article}
\usepackage[left=3.8cm,right=3.8cm]{geometry}
\usepackage{amsmath}
\usepackage{hyperref}
\newcommand{\D}{\mathop{}\!\mathrm{d}}
\newcommand{\I}{\ensuremath \imath}
\renewcommand*{\vec}[1]{\ensuremath{\underline{\boldsymbol{#1}}}}
\newcommand*{\Op}[1]{\ensuremath{\hat{\mathcal{#1}}}}
\newcommand*{\abs}[1]{\ensuremath{\left\lvert#1\right\rvert}}
\newcommand*{\norm}[1]{\ensuremath{\left\lVert#1\right\rVert}}

\begin{document}
\title{Symmetry in the Brillouin zone}
\author{Michael F. Herbst}
\maketitle


\section{Terminology}
\begin{description}
	\item[$\vec{G}$] Vector on the reciprocal lattice $\mathcal{R}^\ast$
	\item[$\vec{k}$] Vector inside the first Brillouin zone
	\item[$\vec{q}$] Reciprocal-space vector $\vec{q} = \vec{k} + \vec{G}$
	\item[$\vec{R}$] Vector on the lattice $\mathcal{R}$
	\item[$\vec{r}$] Vector inside the unit cell
	\item[$\vec{x}$] Real-space vector, $\vec{x} = \vec{r} + \vec{R}$
	\item[$\Omega$] Unit cell / unit cell volume
	\item[$\mathcal{B}$] First Brillouin zone
	\item[$\mathcal{B}_i$] Irreducible Brillouin zone,
		$\mathcal{B}_i \subset \mathcal{B}_r$
\end{description}

\section{Symmetry mapping between $k$-Points}
We are given a set of unitary symmetry operations $\{\tilde{S}_i\}_i$
and associated translations $\{\tilde{\tau}_i\}_i$ on $\mathcal{R}$
on the lattice
\[
	\tilde{S}_i \mathcal{R} + \tilde{\tau}_i = \mathcal{R},
\]
such that for each $\left( S_i, \tau_i \right)$ pair
equivalent classes of atoms are covariant.
In other words if $\mathcal{C}$ is a set of equivalent atomic positions,
then $\tilde{S}_i \mathcal{C} + \tilde{\tau}_i = \mathcal{C}$.

Let $\tilde{S}$, $\tilde{\tau}$ be arbitrary from this set.
We can define a corresponding operator
\[ \Op{U} : L^2_\text{per} \to L^2_\text{per} \]
with action
\[
	\Op{U} : u \mapsto u\left( S^{-1} (x-\tau) \right),
\]
where we identify
\begin{align*}
	&& S^{-1} &\equiv \tilde{S} & -S^{-1}\tau &\equiv \tilde{\tau} \\
	&\Leftrightarrow &  S &= \tilde{S}^{-1} & \tau &= -\tilde{S}^{-1}\tilde{\tau}.
\end{align*}
Assume the potential $V$ to be the sum over radial potentials centred at atoms.
Since the potential term of equivalent atoms is identical we may write
\[
	V(\vec{x}) = \sum_{\mathcal{C}} \sum_{\vec{R} \in \mathcal{C}}
		V_{\mathcal{C}}\big(\norm{\vec{x} - \vec{R}}\big).
\]
Now using $\tilde{S}_i \mathcal{C} + \tilde{\tau}_i = \mathcal{C}$:
\begin{equation}
	\begin{aligned}
		\Op{U} V(\vec{x}) &= \sum_{\mathcal{C}} \sum_{\vec{R} \in \mathcal{C}}
			V_{\mathcal{C}}\big(\norm{S^{-1} (\vec{x} - \tau) - \vec{R}}\big) \\
		&= \sum_{\mathcal{C}} \sum_{\vec{R}' \in \mathcal{C}}
			V_{\mathcal{C}}\big(\norm{S^{-1} (\vec{x} - \tau) - \tilde{S}\vec{R}' - \tilde{\tau}}\big) \\
		&= \sum_{\mathcal{C}} \sum_{\vec{R}' \in \mathcal{C}}
		V_{\mathcal{C}}\big(\norm{S^{-1} (\vec{x} - \tau) - S^{-1}\vec{R}' + S^{-1} \tau}\big) \\
		&= \sum_{\mathcal{C}} \sum_{\vec{R}' \in \mathcal{C}}
		V_{\mathcal{C}}\big(\norm{S^{-1} (\vec{x} - \vec{R}')}\big) \\
		&= \sum_{\mathcal{C}} \sum_{\vec{R}' \in \mathcal{C}}
		V_{\mathcal{C}}\big(\norm{\vec{x} - \vec{R}'}\big) \\
		&= V(\vec{x}),
	\end{aligned}
	\label{eqn:Potential}
\end{equation}
i.e.~the potential is invariant under these symmetry operations.
Since $S$ is unitary we find for a function $u$ in Fouirer space:
\begin{equation}
\begin{aligned}
	\left( \Op{U} u \right)\!(\vec{G})
	&= \int u\!\left( S^{-1} (\vec{x} - \vec{\tau}) \right)
	e^{-\I \vec{G} \cdot \vec{x}} \D \vec{x}\\
	&= \det(S) \int u(\vec{y}) e^{-\I \vec{G} \cdot \tau}
	e^{-\I \vec{G} \cdot (S \vec{y})} \underbrace{\abs{\det(S)}}_{= 1} \D \vec{y} \\
	&= e^{-\I \vec{G} \cdot \tau} \int u(\vec{y})
	e^{-\I (S^{-1} \vec{G}) \cdot \vec{y}} \D \vec{y} \\
	&= e^{-\I \vec{G} \cdot \vec{\tau}}\  u\!\left( S^{-1} \vec{G} \right)
\end{aligned}
\label{eqn:ukFourier}
\end{equation}
Since
\[ \sum_{\vec{G}} \abs{\left( \Op{U} u \right)\!(\vec{G})}^2
	= \sum_{\vec{G}} \abs{u\!\left( S^{-1} \vec{G} \right)}^2
	= \sum_{\vec{G}} \abs{u\!\left(\vec{G} \right)}^2
\]
we have $\norm{\Op{U} u} = \norm{u}$, which makes $\Op{U}$ unitary.
Furthermore
\begin{align*}
	\left( \Op{U} (-\I\nabla + k)^2 u\right)\!(\vec{G})
	&= \left(\Op{U} \left((-\I\nabla + k)^2 u\right)\right)\!(\vec{G}) \\
	&= \left(\Op{U} \left((G + k)^2 u\right)\right)\!(\vec{G}) \\
	&= \alpha \left( S^{-1} \vec{G} + \vec{k} \right)^2
		\  u\!\left( S^{-1} \vec{G} \right)\\
	&= \left( S^{-1} \vec{G} + \vec{k} \right)^2 \left(\Op{U} u\right)\!(\vec{G})
\end{align*}
in other words
\begin{align*}
	\left(\Op{U} (-\I\nabla + k)^2 \Op{U}^\ast \left[\Op{U} u\right]  \right)(\vec{G})
	&= \left( S^{-1} \vec{G} + \vec{k}\right)^2 \left[\Op{U} u\right](\vec{G}) \\
	&= \left( S^{-1} \left(\vec{G} + S\vec{k}\right)\right)^2 \left[\Op{U} u\right](\vec{G}) \\
	&= \left(\vec{G} + S\vec{k}\right)^2 \left[\Op{U} u\right](\vec{G}).
\end{align*}

We saw in \eqref{eqn:Potential} that $V$ is invariant under $\Op{U}$,
which implies for the total Hamiltonian:
\[ \Op{U} H_{\vec{k}} \ \Op{U}^\ast = H_{S\vec{k}}. \]
If the fiber $H_{\vec{k}}$ has eigenfunction $u_k$
with $\varepsilon_k$, then
\[
	H_{S\vec{k}} \ \Op{U} u_k = \Op{U} \varepsilon_k u_k = \varepsilon_k \Op{U} u_k,
\]
i.e. $\Op{U} u_k$ is an eigenfunction of $H_{S\vec{k}}$ with the same
eigenvalue $\varepsilon_k$. We may hence write
\[
	u_{S \vec{k}} = \Op{U} u_{\vec{k}}.
\]
For the total density this implies:
\begin{align*}
	\rho(\vec{x}) &= \sum_{\vec{k} \in \mathcal{B}}
		\underbrace{\sum_{i} f_{\vec{k}}^i \
		u^i_{\vec{k}}(\vec{x}) \ u_{\vec{k}}^{i,\ast}(\vec{x})}_{= \rho_k(\vec{x})} \\
		&= \sum_{\vec{k} \in \mathcal{B}_i}
		\sum_{S_{\vec{k}}, \tau_{\vec{k}}}
		\rho_k\big(S_{\vec{k}}^{-1} (\vec{x} - \vec{\tau_{\vec{k}}})\big) \\
\end{align*}
or in Fourier space
\[ \rho(\vec{G}) = \sum_{\vec{k} \in \mathcal{B}_i} \sum_{S_{\vec{k}}, \tau_{\vec{k}}}
	e^{-\I\vec{G}\cdot\vec{\tau}} \rho_{\vec{k}}(S^{-1}\vec{G})
\]
by analogy with the above derivation in equation \eqref{eqn:ukFourier}.

\end{document}
